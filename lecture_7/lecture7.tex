We don't know what the FTL gives us when we ask for a block which is in trimming list.
HW producer will never tell us how the FTL is made.
Because the FTL is the main component of SSDs and is the thing that makes a drive different from another.
Proprietary information.

In SSDs we have the problem that if we ask for a certain sector, we get from the FTL something which is shown us 

Can we bypass FTL?
    Not by software
    In theory, it is possible to read memory chips themselves by putting them on a bench and access trough it
    But since FTL encrypts/compress/obfuscate data because the way it uses data is a proprietary secret.
    FTL are purposely built to make reverse engineering difficult.
    It would be extremely time consuming because of reverse engineering.
    The process of soldering chips and things may be distructive.
    It is a restriction, you can do it only once and so..
    "Analisi irripetibile". Analysis which destroys the evidence because of the way the analysis works.

    Mobile devices do the same things that FTL does, and the same thing happens ()


Challenges in black box analysis and goals 

    SSD drives in general have the characteristic that FTL can edit data even without any write command to be sent.
    Just need to turn it on.

    Problems with hashes before, after reading 
    
    This also happens with USB sticks 

    
What did they test 
    How trimming impacts drives 
    How garbage collection impacts drives 
    How to recover files considering what wear leveling does and the fact that FTLs may compress and encrypt files 

    they took 3 different drives, they said they were doing the ticked things 
    they disabled caches to perform these experiments
Trimming 

    flow chart slide methodology

Results
    they used different OSs and drives 
    at that moment OSs didn't probably also support 

    Whenever the OS notified appropriately the drive about deleting, all of the files were gone in few seconds
    so the reality is that usually a drive already trimmed itself when connected 
    if trimming worked, data were gone within seconds

    In their tests one of the drives was trimmed on formatting but not on deletion 
    One if unmounted 
    Nowadays most OS will handle trimming correctly and so most of SSD drives would not keep data 
    Since in some cases it don't work correctly it is still important to check them 

Garbage Collection 
    flow chart 
Results 
    they don't do gc 

Erasing Patterns 
    they found out that some drives trimmed by stripes 
    Zanero thinks that it was a faulty drive 
    We cannot exclude that sometimes it fails to trim things so we can still find files 

Compression 
    if you have a drive that performs compression, the throughput is different 
    If a drive performs compression, it is even more unlikely to understand what is stored on the drive when dismounted 

Wear Leveling 
    not interesting because performed by the drive itself 
    The point was to try to test if it created multiple copies of the data 
    The results were that it doesn't

Files Recoverability 
    only when the drive had difects, so it was not correctly trimming 


poi ha chiuso 

