\chapter{Cloud Forensics}
    Cloud computing is a computing-as-a-service paradigm, which has different declinations: IaaS \textit{(Infrastructure as a service)}, PaaS \textit{(Platform as a service)}, SaaS \textit{(Software as a service)}.\\
    We also distinguish between public cloud and private cloud \textit{(company owned)}, but we will consider mostly public ones.\\
    Different issues are present:
    \begin{itemize}
        \item Issues with acquisition and access to evidence
        \item Analysis issues 
        \item Issues with attribution 
        \item Issues of legal status
    \end{itemize}
    \section{Acquisition issues}
        In general, no control is given to the user on hardware and storage space, so investigators cannot really access the metal. This makes traditional acquisition procedures unfeasible for the host, but they are feasible for guests.\\
        It is possible to acquire virtual machines for example.
        Levels of access vary:
        \begin{itemize}
            \item SaaS: the service provider is the only one to have logs/data 
            \item PaaS: the customer may have application log, network log, database log or operating system depending on the content security policy
            \item IaaS: logs until OS level are accessible to customers
        \end{itemize}
        The real problem is that data can not exist sometimes. Instagram pages are not stored, but composed at the moment they're seen. This is hard to understand for court people, and also brings up situations like editorial responsibility.
        \subsection{simple case: web pages}
            What can be possibly go wrong?
            \begin{itemize}
                \item There's dynamic content on the page: how do we capture it, how do we reproduce it in courts? If it's coming from external sites, what is its legal status? If it's an important part, we may be in need to acquire multiple copies to show how they change.
                \item Visualization is different from data 
                \item Attribution: where is physically located the server that deploys the web page? It's not easy to determine where the datacenter is physically located.
            \end{itemize} 
            How do we find the exact location of a web infrastructure with a certain IP address? How do we prove where it is?\\
            It's not an easy question. Expecially if we want to prove where it was in the past. We may be able to prove what the IP address was, but it's very hard to prove where physically the hardware was located.
            The infrastructures are internal, even for who is running the location. Tracking down the specific owner of a machine requires the collaboration of different entities.\\
            Whois allows us to track who owns some parts of the ip address space, who owns a certain domain address. In some cases we need to demonstrate that a certain domain resolves to a certain ip address, this must be proven from multiple points because the DNS resolver may be lie.\\
            Geolocation of hoster: there is no real geolocation of IP addresses. There are good guesses.

