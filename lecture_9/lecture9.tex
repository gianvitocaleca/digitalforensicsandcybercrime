    \section{Relationship with lawyers}
        Lawyers own the choice of defense strategy, they decide what is the legal theory to present to the court. We can give elements but the choice of the theory is left to the lawyer because it has implications that we as engineers are not trained to handle. Also, legal professions have a specific trust relation with the client, they own the relationship with the client, experts should never overshadow this relationship. If you think that the lawyer is hurting the client, you at most can resign.\\
        On the other hand, we don't know what the lawyer knows and vice-versa, we're collaborating. Our goal is not to damage our trust relationship.\\
        The lawyer \textit{(the customer)} is paying us, but, they don't dictate you what to write or to say. The most they can do is to omit something you found if it's not good for the defence strategy. Lawyers may hide things or lie to the court. Witnesses may end up in jail if they lie. Also, in the Italian jurisdiction, some professionals have a right to professional secret \textit{(doctors, lawyers, priests)}, but we as engineers cannot lie if we know something, we end up in jail if asked and we lie or say to don't know.\\
        There may be things that the customer knows that is better that the expert witness don't know, because if asked he might answer.
    \section{Relationship with the customer}
        Any expert witness assesses one of the parts of the judgment, which in civil law are equals, but in criminal law they are the supposed criminal and the law, in Italy they're not equals. By working with the defense you will be considered less credible than the prosecutor witness, especially if a carabiniere.\\
        The mandate of an expert is to find what helps the customer, which is not the same as \textit{helping someone escape law}: by helping any of the parts in a criminal procedure to put forward the best possible evidence in their favor, we're helping the process of determine whether or not the punishment would be applied.\\ 
        \textit{"Process Truth"}, is different from historical truth. If something is not in process truth, against our customer, we must do anything in our power to not let it end up in process truth. 
    \section{Relationship with prosecutors/police}
        Working with the prosecutor doesn't necessarily entail moral superiority. It's still very important to stick to science and facts. It's important to not get your words or thoughts shaped by \textit{justice}.\\
        It's easy to fall into this trap. It's hard to stick to the facts. We have ethics, we want the right thing to happen, but criminal justice is not founded on ethics. So, we cannot act based on our sense of ethics or justice.
    \section{Evaluation: analyzing the documents}
        How does someone read a written report of someone else? In particular, written reports of other expert witnesses and investigators.\\
        Judges read written things, which is the real way in which they work, they'll read our reports, the documents, everything that is written. So what do we look for in others' reports:
        \begin{itemize}
            \item Technical or factual errors (or omissions).
            \item Unclear reasoning, methodologies or descriptions: because if it is unclear enough, it may give us the possibility to say that certain analysis is not credible, and because in court whoever is clearer is more convincing, so if their writing is unclear it is a great opportunity to be clear and explain things in our way.
            \item Suggestive writing: writing which insinuate or suggest something without proof. 
            \item Opinions and hypotheses not clearly distinct from facts and not substantiated: particularly true in criminal law settings.
        \end{itemize}
    \section{Typical technical and factual errors}
        \begin{itemize}
            \item Acquisition:
            \begin{itemize}
                \item Search and seizure: process, chain of custody, seals \dots 
                \item Description of seized/analyzed materials: serial numbers \dots
                \item Hashing/Cloning procedure \textit{(e.g. write blockers)}.
                \item \textbf{Unless this is really important, judge will ignore it}
            \end{itemize}
            \item Analysis 
            \begin{itemize}
                \item Steps where hashing wasn't verified
                \item Use of proprietary programs or with known bugs and/or vulnerabilities
                \item Description of the process
                \item Technical mistakes
            \end{itemize}
        \end{itemize}
    \section{Typical presentation errors}
        \begin{itemize}
            \item The first and most typical that we'll use agains others is the lack of exploration of alternative hypotheses. \textit{(this element could have happened because of this but this do not explain this other thing \dots)}. If there is an alternate hypothesis that explains piecese of evidence taken alone but they do not work in the context usually help the thesis because you already explained and discarded those hypotheses. However, if the other explanation is not credible, I'm not going to help the customer well.\\
                \textit{"Trojan Defense clichè:"} there are trojans and botnets that maybe put images in a computer to make it look like it's owner committed a crime. But things like \textit{"a trojan download images and then perfectly deleted itself"} is not falsifiable. It's a very typical potential defense, so typical that becomes a clichè if not coupled with scientific proof.
            \item Is the presentation neutral or biased?: each expert witness is working for a part, but if while writing a report we put a lot of bias in the report, it end up being less credible to the judge. The less bias we put, the better the job will be. Counter-examples are a powerful tool to deny the credibility of a report. Since the process is a debate, rethoric and counter examples can be a way to diminish the other's arguments. If you have 5 arguments and 3 of them are very strong, bring 3.
            \item Can we find counter-examples for some of the assumptions? Counter-examples are a powerful tool to deny the credibility of a report. Since the process is a debate, rethoric and counter examples can be a way to diminish the other's arguments. If you have 5 arguments and 3 of them are very strong, bring 3.
            \item Are there missing explanations that we can provide in order to shift the understanding of the judge? Whoever is clearer is more credible. Did the counterpart not explained something to the judge? Give them an explanation, establish credibility. The trier of facts do not understand the subject, when you're writing your text think about it. 
        \end{itemize}
    \section{Presentation: writing your report}
    \section{How not to write a report}
    \section{Structure of a report}
        \subsection{Example}
    \section{Testimony as a witness}
    \section{Direct examination}
    \section{Cross examination}