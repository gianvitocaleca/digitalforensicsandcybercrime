    \section{Relationship with lawyers}
        Lawyers own the choice of defense strategy, they decide what is the legal theory to present to the court. We can give elements but the choice of the theory is left to the lawyer because it has implications that we as engineers are not trained to handle. Also, legal professions have a specific trust relation with the client, they own the relationship with the client, experts should never overshadow this relationship. If you think that the lawyer is hurting the client, you at most can resign.\\
        On the other hand, we don't know what the lawyer knows and vice-versa, we're collaborating. Our goal is not to damage our trust relationship.\\
        The lawyer \textit{(the customer)} is paying us, but, they don't dictate you what to write or to say. The most they can do is to omit something you found if it's not good for the defence strategy. Lawyers may hide things or lie to the court. Witnesses may end up in jail if they lie. Also, in the Italian jurisdiction, some professionals have a right to professional secret \textit{(doctors, lawyers, priests)}, but we as engineers cannot lie if we know something, we end up in jail if asked and we lie or say to don't know.\\
        There may be things that the customer knows that is better that the expert witness don't know, because if asked he might answer.
    \section{Relationship with the customer}
        Any expert witness assesses one of the parts of the judgment, which in civil law are equals, but in criminal law they are the supposed criminal and the law, in Italy they're not equals. By working with the defense you will be considered less credible than the prosecutor witness, especially if a carabiniere.\\
        The mandate of an expert is to find what helps the customer, which is not the same as \textit{helping someone escape law}: by helping any of the parts in a criminal procedure to put forward the best possible evidence in their favor, we're helping the process of determine whether or not the punishment would be applied.\\ 
        \textit{"Process Truth"}, is different from historical truth. If something is not in process truth, against our customer, we must do anything in our power to not let it end up in process truth. 
    \section{Relationship with prosecutors/police}
        Working with the prosecutor doesn't necessarily entail moral superiority. It's still very important to stick to science and facts. It's important to not get your words or thoughts shaped by \textit{justice}.\\
        It's easy to fall into this trap. It's hard to stick to the facts. We have ethics, we want the right thing to happen, but criminal justice is not founded on ethics. So, we cannot act based on our sense of ethics or justice.
    \section{Evaluation: analyzing the documents}
        How does someone read a written report of someone else? In particular, written reports of other expert witnesses and investigators.\\
        Judges read written things, which is the real way in which they work, they'll read our reports, the documents, everything that is written. So what do we look for in others' reports:
        \begin{itemize}
            \item Technical or factual errors (or omissions).
            \item Unclear reasoning, methodologies or descriptions: because if it is unclear enough, it may give us the possibility to say that certain analysis is not credible, and because in court whoever is clearer is more convincing, so if their writing is unclear it is a great opportunity to be clear and explain things in our way.
            \item Suggestive writing: writing which insinuate or suggest something without proof. 
            \item Opinions and hypotheses not clearly distinct from facts and not substantiated: particularly true in criminal law settings.
        \end{itemize}
    \section{Typical technical and factual errors}
        \begin{itemize}
            \item Acquisition:
            \begin{itemize}
                \item Search and seizure: process, chain of custody, seals \dots 
                \item Description of seized/analyzed materials: serial numbers \dots
                \item Hashing/Cloning procedure \textit{(e.g. write blockers)}.
                \item \textbf{Unless this is really important, judge will ignore it}
            \end{itemize}
            \item Analysis 
            \begin{itemize}
                \item Steps where hashing wasn't verified
                \item Use of proprietary programs or with known bugs and/or vulnerabilities
                \item Description of the process
                \item Technical mistakes
            \end{itemize}
        \end{itemize}
    \section{Typical presentation errors}
        \begin{itemize}
            \item The first and most typical that we'll use agains others is the lack of exploration of alternative hypotheses. \textit{(this element could have happened because of this but this do not explain this other thing \dots)}. If there is an alternate hypothesis that explains piecese of evidence taken alone but they do not work in the context usually help the thesis because you already explained and discarded those hypotheses. However, if the other explanation is not credible, I'm not going to help the customer well.\\
                \textit{"Trojan Defense clichè:"} there are trojans and botnets that maybe put images in a computer to make it look like it's owner committed a crime. But things like \textit{"a trojan download images and then perfectly deleted itself"} is not falsifiable. It's a very typical potential defense, so typical that becomes a clichè if not coupled with scientific proof.
            \item Is the presentation neutral or biased?: each expert witness is working for a part, but if while writing a report we put a lot of bias in the report, it end up being less credible to the judge. The less bias we put, the better the job will be. Counter-examples are a powerful tool to deny the credibility of a report. Since the process is a debate, rethoric and counter examples can be a way to diminish the other's arguments. If you have 5 arguments and 3 of them are very strong, bring 3.
            \item Can we find counter-examples for some of the assumptions? Counter-examples are a powerful tool to deny the credibility of a report. Since the process is a debate, rethoric and counter examples can be a way to diminish the other's arguments. If you have 5 arguments and 3 of them are very strong, bring 3.
            \item Are there missing explanations that we can provide in order to shift the understanding of the judge? Whoever is clearer is more credible. Did the counterpart not explained something to the judge? Give them an explanation, establish credibility. The trier of facts do not understand the subject, when you're writing your text think about it. 
        \end{itemize}
    \section{Presentation: writing your report}
        Extensive: they need to cover everything. Sometimes your report is the only report.
        If there is a collective of experts, in those cases you'll have more opportunities to comment on things. 
        When you are one, you need to write everything and also to be concise: the ones reading have no time and patience to read something very long.
        In fact, in most cases, if you write a technical report for a judge, he'll look only at the conclusions.
        So, you're writing it trying to be concise and do cover everything, and to also be clear to the judge and the lawyer who don't know about technology, or they would have not called.
        Probably technology is not even in their interests.
        In Italy you probably want to avoid english terms, and if they're not avoidable you need to provide an explanation for them.
        Writing things oversimplyfying them, can make the poor judge to feel treated as a stupid. And since we are technologists this is the part of our profession we don't care about. But in a tribunal the only thing that matters is what you write and how you speak about it.
        Notice that the report is also going to be read by the other experts, which will apply the same checklist above to your report.
        We need to make what we write to matter, we need to map the technical things to the legal matter.
        If something is not relevant, it doesn't map directly to something, it doesn't matter.
    \section{How not to write a report}
        Don't write things like \textit{"it doesn't work"}. Why? Clear the meaning of what you're saying. It may be obvious to you but not to the reader.
        Don't suggest, don't use innuendo. Explicitly say \textit{"this is important because it may lead to that", "it is compatible/incompatible with", "it does/doesn't support it"}. They're all very clear in how much this proves things. We need to prove with facts what we think.\\
        Since we know that most technical things require very in-depth knowledge, we tend to obscure it as demonstration of expertise, and it works with other experts. But if you are too technical in your report, it just will be ignored. The clearest explanation wins when talking with a non-expert. When talking to an expert, the detalied explanation wins.
        You are never going to convince the other expert because he's payed to not to be.
        If you're in the setting of evaluating something with another expert, this is different. But, if the report is for the judge, you need to be clear.\\
        Another typical mistake is to show biases, you do not need to show solidarity with your client, this is expected. It's not like working for the accused person because you don't care about the victim of the crime. You don't need to show in reports the solidarity with the victim. Everyone in the court is equally solidal with the victim.
        No solidarity with the client, he is paying you. You need to analyse everyting in a very cold and factual way.\\
        Don't show excessive deference to the judge.\\
        Don't use sarcasm, however iff you're very sure of yourself, and the other part made an egregious mistake, then a \textit{little} irony.
        Don't use a weak argument if you have a strong one.
    \section{Structure of a report}
        Model it on a scientific paper or report:
        \begin{itemize}
            \item Introduction: what I'm going to say and why it is relevant
            \item Facts
            \item Discussion and analysis: each block with a small introduction saying what you want to explain and a conclusion summarizing what you just describe
            \item Final conclusion: important things you want to convince the judge of. Leave out any doubtful statement.
        \end{itemize}
        Structure it like an obstacle course, each obstacle a little taller than the previous one: make the judge sweat to ignore all of your obstacles.
        \subsection{Example}
            \begin{itemize}
                \item Small introduction of yourself as expert witness
                \item Foreword: I have examined these documents and these evidence sources, I was asked to report on X
                \item TI SEI PERSO UN PO' DI COSE FINO ALLE 13.20 CIRCA
                \item Introduction: Say what we're going to show and how are we going to show that: \textit{"I will show A,B,C,D"}. Since we do present specific parts of evidence, we need to explain to the judge why we're writing about this.
                \item Acquisition issues: if we're working for the defence, we show which are the issues in the acquisition part (hash not computed, \dots). \textit{"this is the lowest obstacle in the obstacle course"}.
                \item On the technical analysis: start with factual errors.
                \begin{itemize}
                    \item even ignoring all of our fundamental issues with the evidence integrity...
                    \item parti mancanti alle 13:27
                \end{itemize}
                \item Conclusions: the only thing most of the judges will consider.\\
                    what have we shown \textit{(evidence not properly acquired... slide 15)} 
                    Don't do their job for them: if you find a mistake, if it is relevant (changes something), we know that this is a weak point, correct it in our minds but then the fix is up to the other side, if they have time, knowledge, understand the problem, they will correct you.\\
                    One strong inevitable phrase to conclude.
            \end{itemize}
    \section{Testimony as a witness}
        In most jurisdictions, the expert witness work do not end with the report. In some of them you may be called as a witness in a tribunal. In Italy, in criminal courts, you'll be called, your report is included in the trial only if you testify. In civil courts there is no need to testimony.\\
        You'll be called to do a sworn testimony. Where you swear that you tell the truth. In Italy no swear. If in specific circumstances you say something that is false, you commit the crime of perjury.\\
        This means that expert witness cannot lie or claim confidentiality or professional secrecy.
        \subsection{Direct examination}
            What is going to happen is that you're being called as a witness by your side.\\
            You start with the "friendly" part of examination, which you may have prepared with the lawyer \textit{(list of questions that they're going to ask)}. In direct examination you're going to be the most clear as possible. Make sure you explain everything to the judge, take your time (not too much). However in Italy, the judge can ask questions at any point in time, so prepare, check previous records of the judge, obviously you can't prepare as for the direct examination.\\
            After direct examination, you stand cross-examination.
        \subsection{Cross examination}
            Examination by the counterpart is less friendly, sometimes downright hostile.
            \begin{itemize}
                \item Prepare: check previous records of the lawyer, prosecutor or judge
                \item Use your report as a shield and as a way to get time and think about the answer
                \item Be curt if you can "yes" or "no", if you cannot, be very complex and difficult to understand
                \item If unexpectedly a question is positive, immediately go back to being extremely clear and helpful
                \item Don't get angry \textit{("Quindi per non essere io quello che finisce in carcere, mi conviene escludere la possibilità di fare questo lavoro")}
                \item Don't be surprised if your competency is called into question
            \end{itemize}
        