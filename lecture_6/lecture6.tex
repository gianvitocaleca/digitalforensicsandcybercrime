\section{Antiforensic techniques}
    They are techniques designed to create confusion in the analyst and/or defeat tools and techniques used by analysts.
    \begin{itemize}
        \item Transient techniques: aim to confuse or mislead the analyst, can be defeated if detected
        \item Definitive techniques: make impossible to recover data
    \end{itemize}
    \subsection{Critical failure points}
        Acquisition \textit{(usage of tools for repeatable cloning and custody)} and identification \textit{(usage of tools for analysis of file systems, data reconstruction and carving)}
        Interfering, it is possible to compromise the process. We talk about transient anti-forensics if we interfere with identification, definitive anti-forensics if we interfere with acquisition.
    \subsection{Timeline tampering (definitive)}
        Files' metadata store also information about modified, accessed, created dates \textit{(MAC)} + Entry Changed (E) value on NTFS.\\
        A typical way to use these data is to take all of the MAC entries of all files in the drive and put them in reverse order starting from the most recent, to display a so-called timeline.\\
        As we go back, since every action overwrites the previous value we just will see the last one, as far we go, old data are shown because they maybe were stored and never touched.\\
        MAC information can be modified to make them appear separated, close, randomized or moving them completely out of scope.\\
        MAC though are not inforced by the OS, they're only declarations. There exist tools just to change these values, those values are not there for security reasons.
        In Unix you can use the touch command to change the thing for example, and there is no track of this, we can have a drive with manipulated access times.\\
        So, as forensic experts we cannot guarantee that these values are correct, and if we use them to support or not support theories we must keep in mind.
    \subsection{Countering file recovery (definitive)}
        File recovery uses data remnants, but some tools can perform \textit{secure deletion}: 
        \begin{itemize}
            \item they can overwrite blocks, fill not allocated parts and/or slack spaces with zeros
            \item they can go on specific blocks and overwrite them
        \end{itemize}
        Encrypted drives can create problems with file recovery, also because they're managed in a way which is different from the usual filesystem's usage of the blocks.\\
        Virtual machines, when dynamically allocating drives \textit{(which are actually files)} have they to shrink if data are deleted and to enlarge when new data is written, often implicitly countering file recovery.
        \subsubsection*{The talk}
            Gutmann in the 80's said that probably it was possible to understand if a zero was written in a sector or a one was, because of the magnetism. In reality it was difficult to do that (impossible) even in the 80s, with nowadays dense disks it is certainly not possible.\\
            But we still have the Gutmann patterns to \textit{"erase"} disk contents.
%sono arrivato qua
    \subsection{Fileless attacks (definitive)}
        Many modern attacks tend to be fileless, with no traces on the disk at all.\\
        For instance, Metasploit's meterpreter is injected in a process' memory space, and gives attacker control without writing anything to disk.
        So, if the machine is off, the evidence is lost, the only thing that can save us is to dump the memory of the machine, if possible, in our acquisition phase.
    \subsection{Filesystem insertion (transient)}
        Data placed where there's no reason to look for them, in particular inside filesystem metadata. Inside a partition table there is space for $\sim$ 32KB of data.\\
        In ext2/3:
        \begin{itemize}
            \item RuneFS can write in bad block inodes
            \item WaffenFS adds a fake ext3 journal in an ext2 partition
            \item KY FS uses directory inodes
            \item Data Mule FS puts data in padding and metadata structures of filesystem ignored by forensic tools
        \end{itemize}
    \subsection{Log analysis ($\sim$ transient)}
        On most machines one of the big sources of info for forensic analysis are logs.\\ 
        Logs tend to be long text files with structured text inside, so they can be manually scrolled or checked by use of regular expressions or scripts.\\
        If attackers can inject stuff in the logs (very likely), they can try to make your scripts fail, or even to exploit them.\\
        Maybe they use an username that contains a linebreak character, the line would be skipped (theoretically example).\\
        Techniques which are transient in the sense that they're trying to confuse the analysis tools. In reality however some of them could be modified in the way in which the log becomes definitive.
    \subsection{Partition table tricsk (transient)}
        \textit{"sci-fi"}\\
        Partitions can be uncorrectly aligned, this may be enough to make a forensic tool to fail. \\
        Adding a number of extended partitions which can be managed by the Operating Systems but not by forensic tools, or a sufficiently large number of logical partitions in an extended one to make the tools fail.\\
        Transient techniques like this work well when for example the drives are a lot and the analyst has no clue of where evidence can be. But, with a small number of drives, if an expert knows what to search for and where to, most of the times he can find the evidence.

