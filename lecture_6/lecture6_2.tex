\chapter{SSD Forensics}
    \section{SSD technology}
        Today, most of the drives are SSDs. SSDs are storage drives made of NAND flash memory chips, which are faster, and were cheaper than an Hard Drive.\\
        By the point of view of the Operating System, there is no difference between \textit{"talking with"} an SSD or an HDD, but the two technologies are slightly different.\\
        NAND flash memory though, has a limited lifetime, and there is the need to manage how to make them to last longer, another difference is that those kind of memory's blocks are only \textbf{fully writable/erasable}, so we don't see slack space in this kind of drives.\\
        When a block is re-written, it has to be \textit{blanked} before, this is a big disadvantage for forensic experts.\\
        What manages SSD behaviour is the FTL controller, which fakes for example to the Operating System the SSD as working as an HDD. It manages a lot of things:
        \begin{itemize}
            \item \textbf{Write caching}: it keeps data in a cache and writes them on the SSD only when needed, to preserve blocks life.
            \item \textbf{Trimming}: it blanks no more useful blocks any time it is idle. (even with write locker connected)
            \item \textbf{Garbage Collection}: they were able to figure out what to delete when Trimming wasn't supported by Operating Systems.
            \item \textbf{Data Compression}: they can store more data in less space, to preserve blocks life.
            \item \textbf{Data Encryption/Obfuscation}
            \item \textbf{Bad Block Handling}
            \item \textbf{Wear Leveling}: they manage blocks in a way to make them be in the same deterioration state.
        \end{itemize}