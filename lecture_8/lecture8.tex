    \section{Analysis issues}
        Virtual machines \textit{(which are the lowest level acquirable things)}, when acquired, have dynamically allocated storage, so their storage will not have remnants of data \textit{(with few exceptions)}. Metadata of VM disappear easily, for instance everytime a snapshot and a restore is done.\\
        Are we going to try? Yes. Is it surely working? No.\\
        If the case is one of hypervisor-level compromises which allowed an attacker to bypass the control of the service provider and jump from one tenant to another, there is no way to investigate this. Usually because they are custom hypervisors with no tools and research. Fortunately it is a very corner case.
    \section{Attribution issues}
        Cloud infrastructures create an additional layer of uncertainty. An IP address with a timestamp will give us a machine, we need to get from there to the specific person, sometimes it is easy \textit{(someone who lives alone)}, sometimes it is hard \textit{(example: families)}.\\
        If you reach the router of a family, and no family member confesses the crime, how do you find the one who committed the crime? It is hard to find specific attribution.\\
        With cloud providers, we have another difficulty: their cooperation. It is not so easy to get that, because most of the times they are not in the same jurisdiction as we are, and we need a \textit{"rogatoria"} to access data from another jurisdiction, it takes time and money.\\
        Since in cloud service providers, there is the need of them to cooperate, this makes investigation of transnational crimes with them involved even harder than normal.
    \section{Legal issues}
        \begin{itemize}
            \item Many types of legal procedures depend on physical geographic location 
            \item Electronic data may span across different locations, or created during a transaction and not existing in any location. We may need to explain in detail where and how data is stored, generated.
            \item Under Budapest convention there is this: ordinarily search and seizure happens physically. A prosecutor issues a warrant for a search \textit{(Mandato di persecuzione)}, and this is done by the Carambineris physically. They introduced the concept of electronic search and seizure: if the thing i need to search is on dropbox, email, .. I am searching the digital possessions, while on a surface level this sound as reasonable, it brings a lot of considerations: in some cases the search and seizure is not really a search, it involves something generated by a transaction, this is documenting. There are specific laws for different procedures, we need to be able to describe the differences in a way which is understandable by lawyers and judges.
            \item Removal of obstacles: when a search warrant is issued in Italy, it usually authorize the police to remove obstacles \textit{(sfondare la porta)}, and this has the effect to make it not a crime because it has been authorized. Giving execution to an order is not a crime in Italy. If the main server to which the access control system needs to be removed is in Norway, it is highly debatable to determine if it is a crime or not. Technical experts must be aware of it. You need to request to Norway's authorities. In digital world it may be not realizable, \textit{(Realizzabile nel senso realizzare mentalmente)}, so you need to be aware.
            \item In our agreements with CSP we need to include the forensic assistance that may be used \textit{(The user ask forensic data to be given to the court)}, because otherwise Microsoft can say "No dati non cielo".
            \item More complex because CSP use each other: at the beginning Dropbox used AWS as a storage provider. Interesting issues.
        \end{itemize}
    \section{Forensically-enabled clouds}
        When we create a contract with a CSP, the fact that it is able to empass with forensic information, may be good for our contract because law may require that my business needs to have a certain kind of accord.\\
        There are also requirements to keep data in a certain location. This may impact optimization. But to be able to offer the service in that location, they do it.\\
        In order to be auditable we may ask to provide snapshots, proofs of past data possessions.\\
        Encryption and key management is important, if certain kinds of keys must be kept by certain organizations, and they store them (?)
    \section{Dual considerations: cloud-enabled forensics}
        A typical laptop today has a 1 or 2TB hard drives. With 100s laptops seized, you need to search in 100sTB of data. Doing this through cloud enabled services it may be useful, but in some cases we're legally obliged to mantain control of the data.\\
        Transfering data on a cloud provider can trigger challenges related to personal data, not all seized people are indagati, we have their data and a duty of care wrt them \textit{(and convicted people)}.\\
        If those data get stolen, indexed because of our error, it may be lead to problems.\\
        As forensic experts, this is a significant challenge.\\
        Also transnational issues because we're performing forensic things in another state.