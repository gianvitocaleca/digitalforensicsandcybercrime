\chapter{Introduction to Digital Forensics}
\iffalse
    Forensics is the application of scientific analysis to try to investigate crimes.
    It embraces a lot of different disciplines, digital forensics is one of them.
    The appropriate translation of forensics is "criminalistica", so "criminalistica digitale".
    We call it "informatica forense"

    Why do we apply scientific methodology? And not  bullshit?
        strong relation with laws, different jurisdiction means different procedures.

        Generally speaking, sources of evidence:
            physical evidence
            witness testimony - because scientists can analyze something, while they don't know about what happened.

            so a computer scientist is allowed to give his opinion in court because they use science and not planets
            science is the main reason because they get listened

        Problem in computer forensics:
            was born in the Usa, a lot of the disciplines has been shaped by the american system.
            we need to be aware that some of the technical things we do, names we use, are related to different courses than others and we must understand if it applies to others
    
    Daubert Standard
        expert witness = someone who has witness because of their experience
        daubert standard is part of the common law in usa (federal rules for testimony)

        A witness who is qualified as an expert by knowledge,
        skill, experience, training, or education may testify in the
        form of an opinion or otherwise if:

        (a) The expert's scientific, technical, or other specialized
        knowledge will help the trier of fact (judge) to understand the
        evidence or to determine a fact in issue;
        (b) The testimony is based on sufficient facts or data; 
        (c) The testimony is the product of reliable principles
        and methods; and
        (d) The expert has reliably applied the principles and
        methods to the facts of the case.

        Reliable principles and methods means scientifically found.
        You need an expert to use scientific method to establish those facts:
        in the us the question was what is scientific method ,
        in italy: "what is scientific?"
            very hard question.
            In order to apply this in course of law we need a basic definition:
            there are two key characteristics of scientifity that we can use:
                galileo and popper 
                scientific = repeatable
                    you actually try to do that and see if it happens
                    and you can repeate the experiment
                scientific = falsifiable
                    whenever we have a statement that is scientific 
                    there is a way to establish if something is scientific or not
                    if we're able to create at least in our mint an example which proves the opposite of our statement, then the statement is scientific 
                    So there are statements that are not scientific. Stefano is a kind person is not a scientific statement. There is no experiment that you can run to disprove that.
                    The question: did this person detained illegal things willingly? Is not falsifiable 

                    Scientific: were the images in cache or in a folder, were they opened multiple times or never
                    so we can answer to correlated questions that are in fact scientific.

        Daubert test for scientific
            Science is not established, is a method to discover things. 
            Or it's not precise enough even if true.

            The generally accepted principles are the ones we are interested into:
                sometimes courts listen to ridiculous theories and judge based on other considerations.
                reasonably judges want experts to bring consensus driven things, how is the judge supposed to use things that are not widely accepted by the community?
                publications and peer reviews are the way to establish if a technique is reliable or not

            rate of error: example when you describe asymmetric encryption for digital signatures
                you can say something like "it's incredibly hard to figure out the key" which means is impossible.
                what the court needs is to estimate the rate of error
                dna testing is very reliable, not perfect but it has an error rate so small that is mostly impossible to make humans collide

            was this research done before this process and applied, or was it conducted in the context of this specific thing?
            if the method was developed earlier and independently, it's more reliable from the point of view of the court. "people lie"

        In italian courts we rely mostly on repeatability and falsifiability.

    WHAT DO WE DO FORENSICS FOR
        Different situations that have different constraints:
        slide 7
        investigation in different contexts 

        depending on which part of the lawsuit you're working for "prosecutor", "judge", "lawyer"
        the things that we can do are different. We always need to square the procedure against who we are and what we are about to do.

        Typical: internal investigations in companies we're outside a court so there are constraints on what is possible to do. They may end up in courts though,

        Even if no prosecution we can use some of the same techniques to analyze the situation
        Also in research 
        
                    
\fi