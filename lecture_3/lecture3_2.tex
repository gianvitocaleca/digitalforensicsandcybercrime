\chapter{Introduction to Digital Forensics}
\section{What does forensics mean?}
    \begin{itemize}
        \item \textbf{Forensics} is the application of scientific analysis to reconstruct evidence.
        \item \textbf{Digital Forensics} is one of the disciplines of forensics, is the application of scientific analysis methods to digital data, computer systems, and network data to reconstruct evidence.
    \end{itemize}
    Forensics is strongly related with laws: different jurisdiction means different procedures.
\section{The Daubert standard (USA)}
    Generally speaking, we can have two sources of evidence:
    \begin{itemize}
        \item physical evidence
        \item eyewitness testimony
    \end{itemize}
    To provide physical evidence, an expert is allowed to give an opinion in court because of the fact that he is an expert, and the use of scientific methods is the reason why he is listened to.\\
    How do we define what an expert is?
    \subsection{The Daubert Standard}
        We define as expert witness someone who has witness because of its own experience.\\
        The Daubert standard define an expert witness as follows:\\
        A witness who is qualified as an expert by knowledge, skill, experience, training, or education may testify in the form of an opinion or otherwise if:
        \begin{itemize}
            \item The expert's scientific, technical, or other specialized knowledge will help the trier of fact\footnote{the judge} to understand the evidence or to determine a fact in issue;
            \item The testimony is based on sufficient facts or data;
            \item The testimony is the product of reliable principles and methods; and
            \item The expert has reliably applied the principles and methods to the facts of the case.
        \end{itemize}
    Reliable principles and methods means that they are scientifically found. You need an expert to use scientific method to establish those facts:
    \subsection{What is scientific? (Italy)}
        \begin{itemize}
            \item Galileo: scientific means repeatable, you actually make an experiment to demonstrate that something can happen
            \item Popper: scientific means falsifiable, if you're able to create, at least in your mind, an example which proves the opposite of a statement, then the statement is scientific.
            \begin{itemize}
                \item Example: \textit{Stefano is a kind person} is not a scientific statement, because it is not falsifiable.
            \end{itemize}
        \end{itemize}
        Some of the questions are not falsifiable, hence not scientific. An example is a criminal which keeps images of kids on his own computer. Did he do it willingly or not? This is not falsifiable.\\
        What is possible to do is to look at the folder to see if it was opened multiple times, look at the file to see if it was in the browser's cache or in a specific folder on the criminal's computer.
    \subsection{Daubert Test for scientific}
    Factors to consider (USA) to establish if something is scientific or not:
    \begin{itemize}
        \item Whether the theory or technique employed by the expert is generally accepted in the scientific community
        \item Whether it has been subjected to peer review and publication
        \item Whether it can be and has been tested
        \item Whether the known or potential rate of error is acceptable; and
        \item Whether the research was conducted independent of the particular litigation or dependent on an intention to provide the proposed testimony.
    \end{itemize}
\section{Example of forensic engagements}
    We do forensics for different situations in different contexts
    \begin{figure}[ht!]
        \centering
        \includegraphics[width=0.5\linewidth]{lecture_3/forensics.png}
    \end{figure}
    \\Depending on which part of the lawsuit you're working for: \textit{("prosecutor", "judge", "lawyer")}, the things that you can do are different.\\
    The procedures has always to be contextualized in what are the purposes and which constraints the purpose has. \\       
    Crimes:
        \begin{itemize}
            \item The first investigated crime is the one that has to do with children. The reason why is that it is one of the easiest to prosecute.
            \item Fraud is the second one. Prosecuting fraud is difficult because there is a large amount of them which are perpetrated by people who live in countries where the laws kind of permit it. It is still denounced because people can get their money back.
            \item Cyber extortion is the crime which consists into stealing personal data to get money from them. Extortion happens with family, work related, revenge cases.
        \end{itemize}
    There exist also a lot of non-cyber crimes which involve digital components:
        \begin{itemize}
            \item Search for traces in digital devices in murder cases
            \item Tracking or geo-localization of mobile devices
        \end{itemize}
    Digital components can be fundamental.
\section{Phases of an investigation (Pollitt)}
    \begin{itemize}
        \item Source acquisition: how we preserve digital evidence
        \item Evidence identification: how we analyze digital evidence 
        \item Evaluation: how we take evidence and pack with the specific case
        \item Presentation: how we put togheter all of this in a court
    \end{itemize}