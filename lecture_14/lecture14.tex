\iffalse
Clustering:
    tries to divide the dataset into clusters (groups). 
    You try to build clusters which satisfy;
        cohesion: maximum omogeneity inside
        separation: heterogeneity between clusters 
    You don't just want to divide ds, you want to find groups, where each group
    has element very similar and each group is dissimilar from the other groups

    Carefully select data 
    More data the better: but avoid excessive amounts of correlated data by applying
    feature selection: if too many features are correlated, clusters may be made using them 

    Clustering for fraud detection
        We want to detect frauds as deviation from the norm
        Find frauds as elements inside your clustering that do not belong to clusters
        or that belong to small clusters far away normal large clusters 
        Important: we're not 100 percent sure that them are frauds, they're just outliers
            so they need to be investigated to understand what they are 
            An efficient unsupervised algo maybe tries to minimize as much as possible the number of false positives

    Distance metric
        to group elements toghether you need to define a similarity measure
        at the end of the day each element of the cluster must be transformed in vectors of numbers 
        to understand if two entities are similar we need to compute distance between them 
        Possible distances:
            euclidean, minkowski or manhattan, mahalanobis (when high variance in dataset)

        Continous vs categorical 
            Continous variables
                euclidean metric, pearson correlation or cosine measure 
            Categorical variables
                binary variables (like if a thing is enabled or not)
                    consider binary features with
                    SMC: simple matching coefficient, compute the number of identical matches between the variable values
                        if you have two binary vectors, it will compute how many matches between the two vectors 
                        1100 1001 --> 2/4
                    Jaccard index: measures the similarity between both (vedi 12.49) per lui conta solo il primo 1, 1/9

                    Use Jaccard index when you have a lot of red flags 
                        if you know only few red flags raised in your dataset you have a situation like;
                            1 0 0 0 0 0 0 0 1 0
                            1 1 0 0 0 0 0 0 1 0
                            SMC will be 8/10
                            Jaccard will be 2/10
                        depending of which kind of similarity we want to pick 
        
        Types of clustering techniques 
            albero slide 

            Hierarchical clustering
                agglomerative strategies vs divisive strategies 
                    aggl. starts by assigning a cluster to each single element 
                        then agglomerates small clusters in greater clusters 
                        until to a single cluster with all elements 
                    divisive does the same thing but in the opposite direction

                The final result is usually in between of the two extremes of the picture at slide 85

                We need a distance to move from step 1 to step 4
                    how to compute the distance between clusters?
                        main problem is that there is no standard way, there are different ways
                        by just selecting one of them you'll have differences in the output

                    Single Linkage: distance by considering the closest elements in the two clusters

                    Complete Linkage: distance by considering the furthest ones 

                    Average Linkage: avg distance between all elements of the two clusters 

                    Centroid Method: elect an element as representative of the cluster (centroid)
                                    compute the distance between centroids 
                                    how to chose the centroid depends on the dataset 

                    Dward: estimate the difference between computing the distance of the two clusters
                            when merged toghether and when separated 12:59

                Number of clusters 
                    which is the best number of clusters?
                    we can estimate it based on the number of data we have 
                    Dendrogram: try to build a tree like diagram that records the sequence of merges based on the distance in which merge toghether or divide the clusters
                        a distanza y abbiamo unito questi 2 cluster, a distanza y questi altri 2 e così via vedi slide 88
                        usually a treshold of 75 percent of the distance sets the number of final clusters
                    Screen plot: keep track of all the distances at which you merge clusters,
                                select a number k when the plot have an elbow
                
                Example of hierarchical clustering 
                    dataset of points which are 2d. Scatter plot 
                    imagine that we decide to apply hierarchical we use single linkage (using euclidean)

                    we put toghether a,b + c,d because they're close toghether 
                        single linkage distance between b and c (nearest)
                        crea (a,b) (c,d) (e) (f) (g)
                        computa tutte le distanze, scopre che b,c sono i piu vicini
                        allora (a,b,c,d) (e) (f) (g)
                        poi cclclclc
                    
                    Con complete linkage we obtain a completely different Dendrogram
                    in this case we'll have two clusters

                    by just using two different measures we obtain different clusters 
                    which are good? probably both.

                    Centroid changes again 
                    Ward changes again 
                
                Conclusions 
                    depending on the things we're going to chose, we get different clusters and clusters number 
                    Advantages:
                        we don't have to specify the number of clustes, we'll se aftwerwards based on similarity measures
                    Disadvantages
                        do not scale very well with large datasets: dendrograms will become almost unreadable + huge amount of different clusters 
                        if from 7 elements 3 clusters, with 3k maybe 1,5k clusters 

                        the interpretation of the clusters is often subjective and depends on the business expert and/or data scientist
           
            Non-hierarchical clustering
                k-means
                Self organizing maps 

                K means 
                    the great difference is that the first step is to decide the number of clusters you want

                    choose the number of clusters 
                    algo picks two random centroids (example k = 2)
                    1st iteration of the algo assigns all the points to one of the two centroids
                    recompute centroids in the two clusters, then redo all 
                    at the end it will converge
                    
                    Limitation
                        if the dataset is easy to cluster ok 
                        otherwise you have to take some decision 
                        you need to find the number of clusters 
                        usually you ask to an expert

                        another possibility is to first run a hierarchical algoritm
                        then pick the number of clusters found by the hierarchichal and use this
                        because k means is designed to find the best output given that k number of clusters while hierarchical was not built for this 

                        Engineering way: try different k and at the end of different iteration compute performance metric, and select the best 

                        When you perform k-means you must not perform it only once, because it depends on the random initialization of the centroids

                        More robust alternative is to use k-mode (using the mode) for categorical variables or k-medoid using the median

                        k means may be biased by the presence of outliers, it will oblige the algo to consider all the points to be part of a cluster, this is true for some versions, other versions with k = 2 will generate 3 clusters with the third being the outliers cluster

                    Slide 105 
                        before, after 
                        a clustering algo always finds a result 

                    Evaluating the results:
                        graphically compare some features of clusters wrt other clusters 
                        compare the distribution of this feature wrt the entire population
                        in red distribution in cluster c1, in blue distribution in the entire population 

                        What we observe in the graph:
                            C1 has a difference in distribution for some features 
                                similar to population in frequency
                                different in recency and monetary
                                the cluster has high monetary value, small recency
                                with a frequency which was similar to population 

                            if you wanted to find clusters of spending patterns this is a good way 

                        Or use SSE (Sum of squared errors) computing the average distance of each point belonging to a cluster 
                        and pick the solution with the smallest RSS 
                        SSE is just for intra-cluster similarity 
                
                SOM: Self organizing maps 
                    unsupervised learning algo that allows to visualise clusters on high dimentional data on a low dimentional grid of neurons 

                    by projecting data on a lower dimentional space organized as a grid of neurons 
                    FFNN with an input and an output layer 

                    Output is usually structured as rectangular/hexagonal 13:32

                    Functioning: slide 111 c'è la figura 
                        for each input assign a weight for each element of the output
                        trying to compute a weighted combination of the input 

                        how will it build the final map?
                            first assigns random weights 
                            then establish a distance metric to optimize

                            compute a distance between the values and each weight 

                            the neuron with the smallest distance is BMU 

                            then update all the weights based on BMU -> move the weights towards the point 
                            try to match the distribution of the original points 
                            This is what happens: slide 115 alle ore 13:37 

                            First advantage:
                                able to automatically cluster data 
                                you can visualzie and interpret the results 
                                    by using the unified distance matrix
                                        add an additional measure to the dataset, superimpose a height Z which corresponds to the avg distance between neurons and neighbors 
                                        lighter area in the left image corresponds to the division between clusters 
                                        riguardalo 13:41
                                    or the component plane 
\fi

                    


