Expert-Based approach:
    Rule-based engine
        Static rules with disadvantages
            they can detect only something seen in the past
            manual and labour intensive to build and mantain
.
..
.
..
.
.
.
.


Lezione nuova 

Fraud management cycle
    Composed by;
        Fraud Detection: apply the fd model on new unseen cases in order to understand 
        Fraud investigation: to understand if it is really fraud or not 
        Fraud Confirmation 
        Feedback loop: Automated Detection Algorithm 
        Fraud Prevention 

Regular update of the model 
    What is required frequency of retraining or updating?
        monitor the system and day by day verify its performances 
        because in general retraining-updating it may be costly, so the frequency depends on the factors in slide 105
        Another possibility is to switch to Reinforcement learning, continously updating the model --> the model is always update
            but the assumption of this approach is that the data are correct and may be learn something fradudolent as non-fraudolent or vice-versa
            There are kinds of attacks that put fraudolent examples in the training set labeled as non fraudolent 

Fraud Analytical Process 
    If you want to build a data driven model you have to follow a sequence of steps that are part of an iterative model 
    Immagine slide 107
    Preprocessing: data selection, cleaning, id, .. transformation most labour intensive, important crytical part 

When you build a model 
    you want it to be able to confirm and detect the same fraud present in the dataset (trivial fraudulent patterns)
    you want it to be able to detect knowledge diamonds = novel knowledeg. They must be able to generalize 

Additional consideration 
    Your alerted fraud must be investigated, the output of the model must be user-friendly 
    It has to be integrated with other applications, part of the company system 
    You must ensure that the model is monitored and updated on a regular basis

Key characteristics of a successful fraud analytics model 
    Statistical accuracy
    slide 117

Stat accuracy:  precision, performance, low false positive, low false negative.
                your model besides being very good at detecting frauds in the train environment 
                must also be able to generalize without overfitting
                noise learned is not present anymore on newly seen data 

                generalize: learn the main trend of the data without fitting the noise 

                general technique to avoid overfitting is train-test split 

Interpretability: you want the model to be a white box model, it means that it must allow to understand why a transaction was labeled as fraudulent
                closed box models are complex, not interpretability, but the community is working to make them interpretable 

Operational Efficiency 
                time 

Fraud Detection Costs 
                fd systems have a cost. It means that to develop one of them you must compute operations which have a cost. Every single time you decide to deploy or not a system you have to go to a cost-benefits balance 

Fraud Management as Risk management 

More Money != more security 

Regulatory Compliance
    depending on the context there may be internal or organization specifica and external regulation that applies to the development and application of a model 
    The fd model should be in line and comply with all applicable regulation and legislation 

    vedi che ha detto cose alle 13.06