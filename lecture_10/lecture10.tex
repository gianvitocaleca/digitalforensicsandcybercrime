What is a fraud?
“Wrongful or criminal deception intended to result in
financial or personal gain” - Oxford Dictionary

“Fraud is an uncommon, well-considered, imperceptibly
concealed, time-evolving and often carefully organized
crime which appears in many types of forms.”
Van Vlasselaer et al.

If you put the two toghether you can better know what a fraud is and the 2nd expecially
tells us what are the five characteristics 

Fraud is a social phenomenon

Why are frauds uncommon?
    Usually the percentage of the frauds are a very small number in a database
    Only a minority of cases concerns fraud of which only a limited number will be known to concern fraud.
    
    They are so:
        difficult to detect
        difficult to learn
        Few history -> errors made when learned

.well-considered and imperceptibly
concealed ..
    Fraudsters try to remain unnoticed and covered.
    ● blend in frauds
    ● not behave different from non-fraudsters
    Fraudsters hide very well by well-considering and
    planning how to precisely commit fraud.
    ● Frauds are not impulsive and unplanned

time evolving
    20 years ago fraud totally different from current one
    because fraud-detection systems learn and improve by examples
    so fraudsters adapt their method to remain undetected

    Fraudsters techniques evolve in time along with or better ahead of fraud detection mechanisms

    A security solution has a cost, if fraudsters are able to adapt in 2 days, then you spent a lot of money for nothing

    Concept drift
        models change in time --- 12.48

carefully organized crime
    Fraudsters do not operate independently, involve complex and organized structures

    Frauds are not isolated events

Cybercriminal ecosystem
    Groups specialized in malwares, infrastructure, money mules, spammers, ... 

    Every single time you're going to think about a financial fraud, you'll see only the final part 

    Initial buyer want to perform fraud: no idea on how to do that but he/she wants 
    groups perform crime and provide the buyer the information of interest he wants.

Why do people commit frauds?
    Basic driver for committing fraud = potential monetary gain or benefit
    Fraud triangle: is a general model that tries to define which are the driver for fraudsters to perform a Fraud
        motivation, opportunity, rationalization
        motivation = usually because there is a need to escape from a tragic situation, or greed 
        opportunity = they know that the system has a low security level or knows an internal person that knows that there is a vulnerability to exploit 
        Rationalization = refers to psychology: usually a fraudster performs fraud when he thinks that what is doing is "not so bad"

A fraud detection and prevention systems aims to reduce as much as possible the opportunities for attackers, make them non-convenient

A lot of fraud categories

Banking credit card frauds
    may happen in two different ways:
        the target is a user 
        the target is the company: convince company to release a cc ... (slide 127)

Insurance 
Corrucption
Counterfaits: create an imitation of an item, presented as genuine
Product warranty: receive item, return back bricks because they knew that the item had 2-3 years until they were checked 
Healthcare fraud 
Telecommunication frauds
Money laundering : usually one of the more efficient crimes 
Click fraud: illegal clicks on advertisements 
Identity Theft 
Tax Evasion
Plagiarism

One of the main vectors to perform frauds is social engineering
    It is the use of influence and persuasion to deceive
    people by convincing them that the social engineer is someone he
    is not, or by manipulation. As a result, the social engineer is able to
    take advantage of people to obtain information with or without the
    use of technology

    "The art, or better yet, science, of skilfully manoeuvring human
    beings to take action in some aspect of their lives”

    “The act of manipulating a person to take an action that
    may or may not be in the target’s best interest. This may
    include obtaining information, gaining access, or getting the
    target to take certain action.”

    Attacker uses human interaction to obtain or compromise
    information by psychologically manipulating a person into
    knowingly or unknowingly giving up this information.

    Essentially 'hacking' into a person to steal valuable
    information from many sources

    A human is easier to compromise wrt an information system.
    Ti sei perso questo pezzo fino alle 13:10


    There is no patch for human stupidity

    Primo criminale poi assunto dal governo

    Frank Abagnale: very famous for crimes committed 
    catch me if you can 


    Another attack is the Phishing attack 
    looking-like legitimate email but 

    Nigerian Prince Scam
        money laundering: they give you the money and then you give them a part 

    Vishing

persone vestite in un certo modo ...
frauds colpiscono anche individui
Frauds impact


31: Anti-Fraud Strategy, fraud prevention and detection 

    Advanced anti-fraud mechanisms:
        Reduce losses due to fraud 
            prevented and detected 
            fraudsters like other criminals, tend to look for the easy way and will look for other, easier opportunities 

    
    Fraud detection: process to recognize or discover frauds once they happen (ex-post approach)
    Fraud prevention: set of techniques used to avoid or reduce frauds (ex-ante approach)

    They are complementary 
    Why:

    slide 35:
    financial services connected to a db, user connected, perform operation, to do it it receives otp ..

    Act to defend the system on the side of financial services 
    because phone + user are compromised 
    try to prevent and detect, automatically analize data in order to 

    Prevention mechanism --> Fraudsters adapt and change behavior -> impact detection power
    by updating detection system -> fraudsters adapt -> impact prevention power 

    Frecce incrociate -> bello 
    you cannot update one without impacting the other 

Approach 
    Analyze data 
        by a local perspective (single user)
        by a global perspective (all the users)

    at the end of the day, all the transaction flagged as frauds are verified by real people 13.41 vedi che ha detto  


Fraud prevention 
    strong customer authentication 
        additional tokens to allow performance of operations 

        examples of classic technologies 
            otp generators
            . 
            .
            to modern technology 

        token generator synchronized to a server 
            but the smartphone may be compromized by an attacker 

        attackers adapt to it 
            sim swapping attack : 
                everybody has a smartphone with an app on it 
                    weaker element was the phone carrier 
                    because if you want to move apps from a smartphone to another
                    they send you a code 

                    1: collect info by you on social nw, other kind of malwares
                    2: contact a phone operator which do not ask for your id
                    3: change operator 
                    bla bla bla c'è l'immagine 

        3d secure 13:49


        