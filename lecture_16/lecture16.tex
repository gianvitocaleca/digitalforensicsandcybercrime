Support Vector Machines
    It basically exploits linear programming.
    Limitation: if you use it to find a decision boundary, you could find multiple solution.
        The constraints make us to find the best solution 

    It wants to find a linear decision boundary that separates the two classes, by finding a support vector 
    Find the extreme points of the classes, called support vectors
    And find the hyperplane that minimizes the distance between the support vectors 

    The cost function is quadratic, so it is convex --> only global minimum.

    In the real world, it exists the non linear separable case.
    Add an error term to consider how much you are misclassifying, and in order
    to move from a nonlinear to a linear space, go to another feature space 
    by applying a transformation of the space using a kernel function.

    Them are not so interpretable, 
    
    Variable selection: backward induction 

    Rule extraction: you can represent a SVM as a NN, and build the rules
    the same way in which you build them for NN.

    Pedagogical Approach: 12:45
    Two-stage models : 12:45

    The procedures explained yesterday are the same to interpret these models.

Ensemble Methods 
    The more complex is the model, the more high is the probability to overfit.
    Main idea is to provide an explanation of the target label by putting togheter different models 
    Combine the results of multiple models 
    Based on the assumption that multiple diverse models capture different features/trends of the dataset

    To be successful, ensemble must be done with models which are sensitive to changes

    They are commonly used with decision trees:

    Bagging (bootstrap aggregating):
        Start by taking B bootstraps from the underlying sample.
        A bootstrap is a sample with replacement.

        Build a classifier for each bootstrap
        For classification: majority voting 
        For regression: average the results

        The number of bootstraps can be decided at the beginning or tuned via 
        an independent validation set.

        If you use bagging with techniques which are robust wrt to data changes the thing will be weak.

    Boosting
        Idea: start from a dataset and assign a weight to each instance of the dataset (uniform at the beginning)
        iteratively re-weight data according to classification error.

        Difficult observations should get more attention

        The final ensemble is a weighted combination of all the individual models 

        You're somehow improving the detection until you find also the difficult ones 

        Problem: you may overfit data if the models keep updating the weights for things you misclassified.

    Random Forest:
        Creates a forest of decision trees.
        Key concept: when you want to build random forests you must ensure the fact that each single tree you select must be 
        as much as possible random.
        The selection of the dataset for the subsample must be random.

        Create as much as diversity in the classifier 
        The higher the diversity, the higher the performances

    Evaluating the ensemble models:
        main disadvantage: they are black box because they are based on random decision trees
        you'll get very different rows for each trees 

        You can calculate the variable importance (12:58)
        just a weight, not an explanation 

Evaluating a Fraud Detection Model
    Two main phases:
        How to split up the dataset 
        Which performance metrics I have to consider 
    
    Splitting of the dataset:
        If large dataset: train, validation, test with no data shared, otherwise you're cheating
            strict separation between them 
            perform stratified sampling (each split must be representative of the dataset)
        
        Small dataset:
            cross validation: try to mitigate the problem of having a small dataset 
                small dataset but not so small 
                divide the dataset in k parts (folds)
                build a model training on k-1 folds and testing in the last one fold 
                then go on until you tested on all folds 
                final performance will be an average of the performance on each of the datasets 
                here you are somehow randomizing the parts to see if the model is really generalizing
            leave-one-out: very small
                train on N-1, test on 1 sample.
            
            Problem:
                cross validation gives us multiple models, 
                    present the results of cross validation 
                    then to provide a model you can build a model based on your entire dataset, re-train 
                    or, build an ensemble of all the models 
                    in the case of LOO, you can pick one at random because the difference is of only one sample.
    
    Performance Metrics 
        confusion matrix 
        based on it you can compute
        classification accuracy
        classification error
        sensitivity, recall, hit rate : how many true positive were you able to correctly detect
        specificity
        precision

        usually accuracy is not a good measure in fraud detection domain
        because frauds are rare, disperse in a huge amount of data 
        they represent 0.1 percent of data 
        if you say that everything is legitimate you will get an accuracy of 99,9

        With an unbalance dataset, accuracy is not good 
        the good ones are F1 measure: armonic mean of precision and recall 
        Matthew Correlation Coefficient
        or other graphical measures (we will see)
        and precision/recall 

        All the performance measures will be dependent on the treshold 
        0= nothing is fraud 
        1= everything is fraud

        Receiving Operating Characteristic Curve 
         13:15
         a good model will be near to the extreme left point of the graph 
        the more you move near the position 
        the more in the diagonal the more random model performance 

        problem: if you have few models to compare, okay 
                to compare very different models, it's difficult 

        Area under The ROC curve 
            solution to the problem (13:17)
        
    Developing predictive models for skewed data sets 
        the datasets will be full of non-positive data examples

        to manage this there are two ways:
            differently weight the samples (higher to fraudulent cases)

            rebalance the dataset by oversampling the minority case or by undersampling the majority case 

            Or decide to use a larger dataset 13:19 circa 

                main challenge of performing under-over sampling is that you have to decide
                    how much to over-under sampling 
                    this will impact the final performance 
                    this methodology must be applied only on the training set and not on the test set 

                    the optimal number of over-under sampling 
                    be much as possible near to the distribution
                        but it really depends on the case 
                            select the one that achieves the best results 
                        bu under-over you introduce bias 

                        empirical procedure 
                            start from bla 
                            poi bla bla 

                        in general it seems that undersampling works better than oversampling 
                
                another technique is: syntetic mynority oversampling technique
                    syntetically generate new samples from the one available
                    identify the k nearest neighbors from the sample you want to replicate
                    and then syntetically generate the new one by average the neighbors
                    so you generated a new one which is slightly different 

        Cost-Sensitive Learning 
            besides interpretability, efficiency, main problem:
                your output if is not in real time it's going to be investigated 
                investigation 
                you can estimate a cost associated to the misclassification,
                in case of misclassification, boooh ultima cosa bro.

                quando dice estimate this cost il cost è quello total ma poi lui ha puntato ai false positives

                the same metric can be used to apply many theoretical procedures 

                Some example of question at the exam.
                
