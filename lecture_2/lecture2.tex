%section: Ransomwares and ransomware attacks
%last subsection: how to get infected
    \subsection{Encryption mechanism: how do ransomware work}
        Once the malware starts running on target computer, it generates a random symmetric key\footnote{used both to encrypt and decrypt information}, usually one per file, and it encrypts each file with it.\\
        The symmetric key itself it's encrypted with a public key, of an asymmetric key pair generated on the server of the group that runs the malware. The private key is stored on the server, and it will released only when the ransom is payed.\\
        Most of this process is automated: when cryptocurrencies are transferred to the criminals' address, the server releases the key associated to it.
        Different keys to allow the attackers to decrypt only a part of the samples to demonstrate that they can do it.
\section{Indirect monetization: Botnets}
    The word botnet comes from the words \textit{robot} and \textit{network}. A botnet is composed by few hundreds to millions of infected devices which run some sort of malware they got by opening an infected email attachment, by plugging an infected USB drive, visiting an infected website \textit{(these are examples of ways to get infected)}\\
    Each botnet has a \textbf{botmaster} who controls and rents it out to perform tasks \textit{(denial of service, spamming, phishing campaigns, crypto mining \dots)}, and his only objective is to earn money by renting them.\\
    The significant challenge with this type of crime is that each bot per-se is not dangerous for the machine itself, but for other machines, and since the cost of cleaning up a machine falls on its owner, some can decide to not do anything about. This cost can be seen as a small fee to be payed by \textit{the community} to get everybody safe. Like a vaccination, some cases of infected machines are not a big trouble for the community, while a lot of them can be really dangerous for everybody.
    \subsection{Rise of the bots}
        Back in the days botnets were used to control IRC chats \textit{(see IRC wars)}. Then instead of using the compromised machines to control the chat, botmasters started to use the chat to control the bots.\\
        In 1999, there was one of the first DDoS attacks, which was against University of Minnesota and used at least 227 bots.\\
        In the 2000s DDoS attacks against high profile websites \textit{(Amazon, CNN, eBay \dots)} got huge media coverage.
\newpage
    \subsection{Geolocalization of botnets command and control}
        \begin{figure}[ht!]
            \centering
            \includegraphics[width=0.6\linewidth]{chart.png}
        \end{figure}
        This chart shows the number of new botnet C\&Cs detected by \textit{Spamhaus} in the second quarter of 2020, and the increase wrt the first one.\\
        By keeping in mind the \textit{observation bias}, we can also introduce another kind of bias which is called \textbf{heatmap effect}: we're biased on the density of the population and on how much data we collect from a certain country.\\
        In this kind of charts we'll always have U.S.A. on top, because there's more penetration in the internet, and also this kind of things is tracked. While we'll have less data from less technological advanced states or \textit{perfectly democratic countries} like Russia or the P.R.C. which can result in lower positions while in reality being the first ones.\\
        Here we see an important increase of C\&C from The Netherlands:
        \begin{itemize}
            \item it is possible that \textbf{something is on}
            \item or simply that some Netherlands organization decided to participate in this data collection feeding a lot of \textit{new} data
        \end{itemize}
        Funny thing about Russia and P.R.C: Russian people hack russian computers too, while P.R.C citizens don't hack into their fellows machines.
\newpage
    \subsection{Type of (botnet) malware families}
        \begin{figure}[ht!]
            \centering
            \includegraphics[width=0.6\linewidth]{families.png}
        \end{figure}
        Every botnet in general can be used to do any sort of things, but they're used to do only one of them:
        \begin{itemize}
            \item \textbf{Credential Stealers:} used to steal sensitive credentials.
            \item \textbf{Banking Trojans:} credential stealers specifically designed to perform stealing of bank accounts information. \textit{(Gozi is the most common one, Zeus the most important one)}
            \item \textbf{Remote Access Tools:} basic botnet oriented malware, which allows to control a computer in order to perform whatever.
            \item \textbf{Loaders:} specifically designed to allow a botmaster to load a program of whatever sort on a computer. Maybe a client ask you to install a certain malware on a lot of computers, and you can do it with a loader.
        \end{itemize}
        We talk about families because there are criminal groups that only develop their source code, and who performs the attacks buys the code and personalizes it for the specific purpose they need.\\
        So we can consider three businesses: how to develop them, how to configure them, how to use them.\\
        There is a market for services related to malwares and cyber attacks \textit{(cybercrime as a service, underground market...)} structured around the needs of cyber criminals. This market is fueled by the money that these schemes make. Some of these money pays for the tools used to make that money.
\newpage
\section{The cybercrime ecosystem}
    The status quo consists in \textbf{organized groups} performing \textbf{various activities:}
    \begin{itemize}
        \item Development and procurement of exploits
        \item Site infection
        \item Victim monitoring
        \item Selling \textit{exploit kits}
    \end{itemize}
    \begin{figure}[ht!]
        \centering
        \includegraphics[width=0.6\linewidth]{ecosystem.png}
    \end{figure}
    These ecosystems can exists because some of the activites done are not illegal: developing exploits per-se is not illegal, selling one is not illegal, the usage of them may be or may be not illegal.
    Even the services related to the configuration of malware are not illegal, while execution and operation may be.
    If you're sufficiently shielded and you run these \textit{businesses} in countries that cannot persecute you, we can even know your name and surname but you cannot be arrested.
    \begin{itemize}
        \item Developers write the source code for malwares with the help of packers and exploits developers
        \item Crime service enablers:
        \begin{itemize}
            \item Quality assurance makes sure that antiviruses don't detect their software
            \item Bulletproof hosting is a kind of \textit{close-an-eye} hosting, \textit{(e.g. russina business networks)} with permit certain suspect activities over their infrastructure, they're sort of borderline organization with lot of regular customers, and some bad ones
        \end{itemize}
    \end{itemize}
    \subsection{Identity sales}
        People's identities are sold on the black market. They are worth because they can be used for fraud, to open bank accounts used for money laudering for example.
        The more they're useful, the more they are worth.
        Worth 50 dollars for you that sell it, the one who pays is going to use them for fraud which is worth more.
        The black market is fueled from an enormous amount of money that people stole.
    \subsection{Drive by download}
        \begin{figure}[ht!]
            \centering
            \includegraphics[width=0.6\linewidth]{drivebydownload.png}
        \end{figure}
        This is another way in which malwares are installed:\\
        an exploit breaks into your browser and executes code on your machine. This requires your browser to be vulnerable and for you to visit a compromised website with the exploit running \textit{(streaming, cracks of programs websites,\dots)}.
        The most compromised are \textit{factually illegal websites}: you're not scared of the strange url if you are in need to see the last movie in streaming, download the expensive program's crack, this is in contrast with a normal situation like going to amazon to buy shoes.\\
        But, \textit{traps} can be also present in legitimate websites.\\
        What actually happens is that the users endup in a series of redirects, called \textbf{redirect chain}, which will endup in a visit to a website running what is called \textbf{exploitation pack} for your browser.
    \subsection{Exploitation kits sales}
        This kind of kits is sold in the dark web by well known organizations, take \textit{Blackhole} as example:\\
        they were buying exploits from developers to earn 10s of millions per month renting them.\\
        Their boss, surprisingly called Dmitry “Paunch” Fedotov, was arrested in 2013 after years of running the website.
    \subsection{Monetization on the dark web}
        Monetization takes a lot of forms, one of them is stealing credit card numbers to sell them for a relatively small price.\\
        The hackers get \textit{easy and fast} money, while the buyers need to use the real money on the cards to buy expensive objects or to start frauds because they can be easily refunded. \textit{(apple computers, ethnic travel example..)}.
    \subsection{Cybercrime and perception}
        Online, you detach people from the picture. It's actually easier to commit crimes, it's easier to crack a program instead of stealing a car, lot of cyber criminals would never be criminals in the real world.
        But when you do cybercrime, your perspective is different: you have another perception of what you're doing, that's the reason because crime happens more online.\\
        Ethical people understand the consequences of their actions, and may decide to not do them, but someone can not care about what can happen as a consequence of their actions and perform them anyways:\\
        in the Jamal Khashoggi case, the exploiter which wrote the code for the spyware that ended up to make him killed, would never kill someone. But the consequences of his actions did.
    \subsection{Money mules and money laundering}
        Most cybercrimes end up with a digital form of money, which crimials want to bring in a physiscal form, completely disconnected from what they did in principle.
        \begin{itemize}
            \item They can pay invoices to companies somewhere in the world, which in a certain way makes sure that the money come back clean. \textit{(traditional way)}
            \item They can make use of \textbf{money mules}
        \end{itemize}
        Money mules are intermediary people which by purpose or not make earnings coming from cybercrime clean money.\\
        Accomplished ones may be people that have nothing to lose \textit{(prejudiced, poor people \dots)}, they just open bank accounts by their names and get the money transferred and then withdrawn and handed to criminals.\\
        Some of them may also open accounts by somebody else's name using identities that may be stolen with an attack and bought online. They're most difficult to catch because the Police needs to be actually there while they are withdrawing money.
        \\Unaccomplished ones may be people which are fooled by things like the Nigerian prince scam: they get the money on their bank account, send 70\% of them to criminals in packages or by moneytransfer, and keep 30\%. Most of the cases they get arrested and have to pay back all the money, also the ones they don't own no more.\\
        Buying and selling of goods \textit{(ricettazione)}, videogames curriencies, \dots, are other ways to perform money laundering.
\chapter{Cryptocurrencies: abuses and forensics}
    We talk about cryptocurrencies in the way they are used in cybercrime.
    \section{Bitcoin and blockchain}
        Bitcoin is an attempt to create \textbf{electronic cash}: a digital currency that has lots of properties that physical money have, one of them is \textbf{no central authority}.\\
        What Satoshi Nakamoto wanted to have is an electronic distributed ledger\footnote{ledger=registro}.\\
        Having no central authority is a really hard problem to solve \textit{(byzantine consensus)}, he did it with \textbf{the blockchain}.\\
        The blockchain is a \textbf{shared, append-only, trustable} ledger of all coins transactions. The limits of distributed consensus defined in the Byzantine Problem and CAP Theorem are solved using the technique of \textbf{proof-of-work}.
    \section{Wallet and addresses}
        A wallet is the software that allows to manage and store the \textbf{public} and \textbf{private} keys for each of the user's bitcoin addresses. To create and sign transactions, track the balance.\\
        A bitcoin address is an alphanumeric string which identifies a \textit{point} where you can send bitcoin to, and where they can be sent from.
        In order to know how many Bitcoin are \textit{stored} in that key you need to go to the origin of all the transactions and track the flow of them to understand how much of them are there now. This computations is not really efficient but can run without central authority, so the only reason for using a blockchain is if it is really so important to get a way out of a central authority.
\iffalse
    bitcoin address is alphanumeric string which identifies a point where you can send bitcoin
    and where them can start from.
    In order to know how much btc are in that key you need to go to the origin of all the transactions and track the flow of them to understand how much of them are there now.

    Not really efficient but can run without central authority 
    The only reason for using a blockchain is if it is really so important to get a way out of a central authority
    it is the less efficient thing ever but it has a way out of central authority.
    for any other reason it doesnt' make sense.

    WE WANT TO DO TRANSACTION ON A LEDGER WITH NO AUTHORITY, WHE USE THIS SHITTY TECHNOLOGY
SLIDE 6
    as soon the transaction is on the immutable ledger the money are on the other address.
    How do we ensure that this is the only transaction done with that bitcoin 
    we need to have a consolidated immutable ledger
    with a central authority this would be easy.
    how do we agree on a ledger which cannot be back modified?

    slide 7
        certain btc associated with your public key
        joe signs a transaction which moves them to alices' wallet.

        I need to have an history that cannot be modified
            Joe could unsign the key. How do we do?

        Transactions are put in a block, block linked in a so-called block chain (linked list)
        we want this list to be immutable

        Mining:
            Solves 2 problems:
                ensuring that the list cannot change 
                generate bitcoin

                we seal the block and get a little btc in reward, and when we seal the block we tell everyone were we want to send the reward

                mining is a very computational demanding activity
                    hash function 
                    collect transaction, put them in the block, put there a destination for a reward
                    bruteforce the block in a way that the hash of the block has a certain number of zeros to start with.
                    This is difficult, the more zeros at the beginning, the more difficult for a miner to find the right combination.
                    Miners are using computational power to trying to find the right solution, once they find it they publish the solution.
                    At this point the other miners, look at the transaction that are not in a block yet, try to put them in the next block and try again.
                    The more computational power you put, the faster you will get to the solution.

                    The difficulty increases more and more, the reward got smaller and smaller but btc value gets larger and larger.

                    block = set of transactions + reward location which a sha256 with a certain number of zeros.

                    After I found a solution, the new block must be linked to this one ... mancaun pezzo 13.56 sul next hash 


                    Sooner or later two entities get a solution at the same time or more or less. What happens?
                    For some people chain ends with block a and some with block b, do i append to a or to b?
                        what happens is a fork
                        slide 13,14 ...  
                            alternative ending of the chain
                            some nodes see the 1st, some the 2nd.
                            you start from there and search the next one.
                            you can see now two different blockchains.
                            sooner or later someone finds a solution
                                who mines on the green blockchain fount the purple solution
                                    the rule now is that the true blockchain is the longest one.
                                        purple block is one ahead, i'm no more in the true one. 
                                        it doesn't make more sense to try again the red one.
                                        give up and go to the shortest.
                                    
                            This is done because in order to revert a payment you'd need to go back on your block, go ahead mining until your blockchain is now the longest one
                            means that you compete with all the others.

                            Blockchain solves the problem of central authority by making people to do useless thing 





\fi