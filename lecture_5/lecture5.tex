    \subsection{Alternative 2: Target powered on}
        For many reasons can happen that the target machine is turned on:
        \begin{itemize}
            \item Maybe it is running a critical service for the company we're working for
            \item Maybe it was seized while turned on
        \end{itemize}
        Working with a powered on machine is called \textbf{live forensics}.
        Can we turn it off? Probably not if the machine is a critical one. Should we turn it off? Maybe not if we're trying to do live analyisis of an intruder. We want to exclude an intruder? We can disconnect the network.\\
        Since the machine is turned on, our actions are going to modify its state, our commands, even turning the machine off will change it, because of the operating system's operations.\\
        Now the prospective changed, before we thought of evidence as immutable as possible, now we don't.\\
        Since our actions are going to tamper with the state of the system, we need to \textbf{document what we did} \textit{(to preserve chain of custody (USA))}, here we don't have the safeguard of the hash.
        Work in volatility order:
        \begin{itemize}
            \item Dumping the memory may be useful, for example most of the break-in attacks are file-less.
            \item So is saving runtime information: network, process information, etc\dots
            \item Consider encryption before turning the machine off \textit{(maybe the device can become unusable after)}
            \item Finally, disk acquisition
        \end{itemize}
        If possible, perform the acquisition without a shutdown, if not, if the machine has to be reboot after, shut it down properly. If it is not crytical it is possible to pull the plug to not tamper the disk.
        \subsubsection*{Useful commands}
        \begin{itemize}
            \item Network data: \textit{"ifconfig -a ; netstat -anp ; route -n ; arp"}
            \item Process data (to store process information): \textit{"ps aux ; lsof file"}
            \item Users data: \textit{"who; last; lastlog"}
            \item Memory acquisition: \textit{"Mantech mdd, win32dd, Mandiant Memoryze"}
        \end{itemize}
\subsection{Alternate 3: live network analysis}
    In enterprise environments if we want to observe an attacker \textit{live} whithout him being scared of our actions, we can use two observation points that are outside of the machine: logs and network traffic \textit{(we'll have a separate class)}
\subsection{New Challenges (separate classes)}
    \begin{itemize}
        \item Forensics in cloud environments
        \item Mobile forensics
        \item SSD drives work different than magnetic drives
    \end{itemize}