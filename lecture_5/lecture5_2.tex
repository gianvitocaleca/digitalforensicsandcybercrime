\chapter{Identification}
    \textit{understanding DB structure, understanding source code, reverse engineering}
    Identification means to perform actions that fall anywhere in computer science.
    We look at either the methodology and specific forensics cases.

    Tipically we use Linux:
    \begin{itemize}
        \item extensive native file system support
        \item we can just dump a disk image as a file with linux, and we can mount the copy as if it were a drive, also read only to prevent writing and keep the copies not touched.
        \item common to use virtualized windows machines to execute some utilities, software available only on winzozz
    \end{itemize}
    There is an exception, several forensics tools run under windows and replicate a lot of these functionalities e.g. xways 

    Nobody uses windows by itself. It must be confined, it stinks.
    It automatically mounts drives and writes on them.
    No hot swap devices, no support for every file system
    Best is to use linux with windows virtual machine.
    Typical way to share file use vmware tools or network share.
    Samba-SMB 
    Keep in mind that if you share something in these ways, this sharing is file level, any utility that analize drives cannot work in this way.
    To use a windows utility to analyze drive, you need to mount the copy as a drive for windows, usually with virtualization software.
    In these cases, you can use non-persistent mode of vmware/virtualbox: map a device so that any write get recorded on the side, not actually changing the drive.

    CONSIDERATIONS THAT WE NEED TO TAKE IN OUR APPROACHES TO ANALYSIS 
        Basic consideration goes back to our definition of scientific = repeatable 

        Experiment: without explanation of how the experiment works, it doesn't carry any meaning 
                    examples: machine does dna matching (two samples, outputs a probability). The only way for this to be acceptable as scientific is by showing that the machine under reasonable circumstances performs the operation, we can assess how well, steps are always the same, once we've opened the machine, closed and sealed it, we could at almost admit the results. We should be acble to tell if the machine is still working correctly 
        
        Use open source sw or you need to be able to show me that the tool outputs can be reproduced with another experiment
        Law enforcement only tools like kid's images thing 
        You have to show me that the same can be extracted with a non proprietary tool.
        proprietary software have open source access for judges or people related 

    DURING ANALYSIS we WORK ON EVERYTHING
        We may need to apply many techniques from computer science.
        The reason why people goes to a forensic expert 
            being able of doing analysis 
            competence: things he knows about cs in general.

            good forensics expert maybe cannot be competent in processor design 
            call someone else to work with (La Sciuto)

    RECOVERY OF DELETED DATA 
        Data/evidence voluntarily or unvoluntarily deleted because time passed, by the OS, by formatting drive, by a fault damaged drive.
        In many of these cases, we may be able to reconstruct all or part of the information 
        ora vediamo il file system di unix 

            Is a way to orgnaize space on a drive to store information. Archive like thing 
            The basic idea is to have units of information= files, to put on the storage drive, so to have an index of what is stored and where.
            If the index is there I can go to the place where things are stored, with no index i may not reach them but they can still be stored.
            inodes keep pointers to blocks (metadata)
            in unix you have a series of ten direct block entries that tell there is this content on the drive and it belongs to the driv ease
            pointers* that point to data blocks 
            i have metadata with pointers 

        when we delete a file the operating system just marks the file as deleted and not show it no more 
        these things will go away at random and independently from each others 
        sometimes it happens that the os reconstructs the tree of the filesystem, it doesn't care about deleted files, at this point the metadata flagged as deleted disappears.
        Independently from this, sometimes when the os need the space of a deleted datablock it takes it.
        those things happen at random, and in the case of blocks they won't necessarily happen all at the same time.
        statistically on a large hard drive there is a good possibility that metadata goes away before data blocks.

        DISK GEOMETRY
            tracks, cylinders
            when you read and write, the minimum portion of the track is called sector
            we still use this terminology to refer at the "geometry" of the drive.
            What we care about is the sector 
            The os can not really work on a sector, because of different sector sizes 
            filesystems are made to work on clusters of sectors 
            mimmo NTFS created clusters of 4kb 
            independently from the sector dimension on the drive.
            
            OS store files cluster by cluster. You store a 5kb file on a filesystem of size 4k you use 4k + 1k/5k of another sector 

            Interesting effect is that the file is blocked by clusters, written by sectors. This file spans 7 sectors, (image). But the drive will not be overwritten by data because the data we wrote is less.
            Slack space 
            In the slack space we know there are remainances of old files.
            Btw some files get stored once and stay there forever. Example dll 
            Shared libraries 
            What happens if i save a dll on top of a file that was there before?
            The slack space is there with the dll forever.
            Years later you can find files deleted years before.
            Disadvantage is that they are a small portion.
            Less than 4k of an email, webpage could be useful to retrieve
            of an image, video, big files less useful.

            Some filetypes need the complete file to be interpreted.
            Small fragment of encrypted file is useless.

            disconnected image 

            Sectors are one after the other (slide 12), if we place a file on a drive there is a good possibility that all the sectors of a file are one after the other.
            Put ourselves in the situation in which metadata are lost but data still there.
            I can take the entire drive, scan it as a string of bytes and seek 
            issues: fragmentation or other things he said 13:57

            typical though by seeking we can seek for specific things that have headers and footers and collect them.
            Some of them will be broken, some of them will have been overwritten in some places 
            Some of them can be weird 
            Some of them will be perfect. 

            The issues are what happens if the file is fragmented, it was a big problem on small drives, now in most cases drives are so large to not fragment 
            and most of os try to not break files for performance reasons.
            If files are broken usually they're broken in two 
            There are some techniques to find fragmented portions of a file 

            Another issues is with encrypted, compressed files. In those cases we really need the file to be complete 
            Headerless files because they were taken away header or footer
            compressed/encrypted headerless are not approachable.

            slide 13 tools 
                photorec is a carver but it works better with pictures and videos 
                gpart to recover damaged partition tables carver that looks for partitions
                testdisk to recover 
            


    
