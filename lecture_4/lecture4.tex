% ultima cosa \section{Example of forensic engagements}
\chapter{Acquisition}
    Forensics was born in the USA, it borders with law so it was developed with US laws in mind.\\
    The legal system of USA and the EUs ones are extremely different, and so is the court approach:
    \begin{itemize}
        \item In the U.S. most of the cases are tried by juries of peers, the judge works to make the court work. In fact, the judge is the one to decide if evidece is admissible or not. If not admissible, that could not be talked about from jury and lawyers or taken into account. Admissibility of scientific evidece is completely based on the concept of \textbf{chain of custody}: \textit{tracking where the evidence was taken, who was in custody, where was it stored, who analyzed it, what was done to it}. If that is broken, evidence becomes inadmissible.
        \item In Italy is the judge the one who takes the decision. The jury exists only in Corte d'Assise and it is made by people extracted from a certain set. Is the judge to evaluate the admissibility of the evidence, it is inadmissible only if it was taken in violation of laws.
    \end{itemize}
    By the way, the Counsil of Europe has an international law agreement called \textbf{Convention of Budapest on cybercrime}, it was written in 1999, and was incorporated in the Italian law in 2008. Also international standards from ISO/IEC can be applied.
\section{Brittleness of digital evidence}
    Digital evidence is \textbf{brittle}\footnote{fragile}, it is because if it is modified there is no way to tell. In other words, it is not \textbf{tamper evident}\footnote{non mostra segni di manomissione anche se manomessa}. This means that there is no way to say if the chain of custody was violated.\\
    Digital evidence can also be fake created:
    \begin{itemize}
        \item By changing for example the clock of a computer, it is possible to create a fake file which was modified in a different time, and there is no way to figure out if that happened.
    \end{itemize}
    We need an entire process of acquisition to create ways to make digital evidence as far as possible tamper evident, there is a need to ensure:
    \begin{itemize}
        \item Legal compliance: evidence must comply with the laws.
        \item Ethical behavior from all parties: even the police can act unhetical
        \item Detection of errors in good faith 
        \item Detection of natural decay
    \end{itemize}
\section{The usage of hashes in digital forensics}
    Hash function are used to record the state of an object at a given step of acquisiton. It is constantly checked to ensure authenticity and non-tampered state of that at any further phase of acquisition. They are used to \textit{"freeze"} the crime scene.
    \begin{itemize}
        \item We don't know what happened before that first step
        \item We can only tell if something has been tampered, but not what has been tampered. We also cannot restore data if tampered.
        \item Hashes are used to prove that something has been modified, but cannot tell what.
    \end{itemize}   
    Since evidence is going to be stored on a certain media, hashes must be put in another place \textit{(e.g. writing them on a physical register is the most frequent, or use digital signatures)}.
\iffalse
SW AND HW FOR ACQUISITION
    HW: write blockers (we'll talk about)
        embedded devices 
        or
        SW: linux supports most extensive file system + una cosa con il mounting (we'll see)

BITSTREAM IMAGES:
    usually we acquire a machine or a drive, what we want to acquire is a bitstream image of the drive: a bit-by-bit copy 
    the basic reason is that if we only copy the visible content we leave behind all the possible info contained into unallocated space 
    %è entrato un piccione
    this may be different with special cases: raid drives, virtual drives,..  

    we call it "freezing a drive"
    forensic clone/copy/acquisition 

BASIC PROCEDURE OF ACQUISITION:
    GUARDA LA SLIDE 
    + immagine di write locker che è a bridge that connects to sata port blocking the writing signals.
    examples of commands are from Linux 
    comando 1: pipeline con utility to sha256sum the sda 
        option noerror= keep going even if the drive generated an error 

    comando 2: copy the source, give a name to the file which will contain the bit by bit copy of the drive 
    most filesystem have a size limit for file dimension

    if hashes are different means that the copy didn't happen correctly.
    or maybe you tampered with the drive 
    maybe damaged block reads differently or destroys when you read it 
    the drive broken between the two times 
    doesn't happen often but it can happen 
    these operations may alter evidence, this has consequences 

    by comparing the three hashes we can find what exactly happened when something goes wrong. 

    old hashes maybe because previous copies when sha1 or md5 were used 

CHALLENGES: TIME 
    a tipical hdd (traditional one) at a tipical transfer speed and if also write blocker is on usb.
    for a 2tb size drive it's going to take even hours 

    with the process seen before is 3 times the time.

    but there are tools like dcfldd that will do the things in parallel 

ACTUAL way
    use embedded duplicators

CHALLENGE: size 
    filesystem limits to 4tb mostly for the size of a file. So you need to break it up.
    If you work for an extensive investigation, 10s 100s laptops means 10s, 100s terabytes.
    need for storage area networks storage facilities.

    if you're not in the lab, one of the few ways to manage things is to transmit data trough the network.
    a way can be to use netcat
    put it listening to a port 
    and run dd piping to netcat sending 

CHALLENGE: encryption
    many machines use encryption, mac, windows, mobiles 
    in some cases the keys for decription are on the motherboard of the computer 
    the image is going to be useless without the computer 

    if keys are stored on board it can be a problem even if the person wants to provide the key because(?)

these were the basic procedures 

COMPLICATED CASES, ALTERNATIVE PROCEDURES 
    it is hard to disconnect the drive or it is a bad idea (examples: hardware based encryption, RAID devices if it is a hardware RAID the implementation depends on the hardware card, if we take out the drives and copy they images we don't know how put them togheter, iMac, not standard drive connectors)
    we need to boot that hardware
    if you boot the os on the machine, the drive is changed (swap file, temporary file) you never want to boot the operating system
    boot from another boot media, live linux 
        do it with great care, different machines have different ways to do boot from different drives 
        you don't want to accidentally boot the real operating system.
        if the drive is disconnectible, try to boot from the external device, once you know how to do it then reconnect it 
        sometimes you need to do it properly the 1st time 

        you cannot use any distribution, swap partitions on the drive marked for being linux swap, the thing will use it
        use certain distributions like tsurugi, backbox.
\fi



